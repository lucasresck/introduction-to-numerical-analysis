\documentclass{article}
\usepackage[T1]{fontenc}
\usepackage[utf8]{inputenc}
\usepackage[portuguese]{babel}

\title{Lista 3 \\
\large Introdução à Análise Numérica \\
Interpolação}
\author{Lucas Emanuel Resck Domingues}
\date{Outubro de 2020}

\usepackage{natbib}
\usepackage{graphicx}
\usepackage{amsmath}
\usepackage{listings}
\usepackage{multirow}
\usepackage{hyperref}

\lstset{columns=fullflexible}

%Hyperlinks
\usepackage{hyperref}
\hypersetup{
    colorlinks=true,
    allcolors=,  % Nothing change colors
    urlcolor=blue  % URL changes color
}

\begin{document}

    \maketitle

    \begin{enumerate}
        \item[2.] Vamos utilizar o método das diferenças de Newton,
            encontrar os coeficientes $c_n$ e finalmente encontrar $P(k)$.

            $P$ é um polinômio de grau $n$, e sabemos que ele passa
            pelos pontos

            $$(1, -1), (2, -2), \cdots, (n, -n), (0, (-1)^n)$$

            sendo o último ponto justificado por $P(0) = (-1)^n$
            (termo independente).
            Então o polinômio escrito na forma do método das diferenças fica

            \begin{align*}
                P(x) &= c_0 + c_1 (x - x_1) + \cdots + c_n (x - x_1)\cdots(x - x_n) \\
                &= c_0 + c_1 (x - 1) + \cdots + c_n (x - 1)\cdots(x - n) \\
            \end{align*}

            A Tabela \ref{tab:diff} mostra o cálculo dos coeficientes $c_i$
            pelo método das diferenças de Newton.

            % https://www.tablesgenerator.com/
            
            \begin{table}[!h]
                \centering
                \begin{tabular}{c|c|c|c|c|l}
                    x &
                    $y=\Delta^1$ &
                    $\Delta^2$ &
                    $\Delta^3$ &
                    $\Delta^4$ &
                    \multirow{18}{*}{$\cdots$} \\ \cline{1-5}
                    \multirow{2}{*}{1} &
                    \multirow{2}{*}{$\mathbf{-1}$} &
                    &
                    &
                    &
                    \\
                    &
                    &
                    \multirow{2}{*}{$\mathbf{-1}$} &
                    &
                    &
                    \\
                    \multirow{2}{*}{2} &
                    \multirow{2}{*}{$-2$} &
                    &
                    \multirow{2}{*}{$\mathbf{0}$} &
                    &
                    \\
                    &
                    &
                    \multirow{2}{*}{$-1$} &
                    &
                    \multirow{2}{*}{$\mathbf{0}$} &
                    \\
                    \multirow{2}{*}{3} &
                    \multirow{2}{*}{$-3$} &
                    &
                    \multirow{2}{*}{0} &
                    &
                    \\
                    &
                    &
                    \multirow{2}{*}{$-1$} &
                    &
                    \multirow{2}{*}{0} &
                    \\
                    \multirow{2}{*}{4} &
                    \multirow{2}{*}{$-4$} &
                    &
                    \multirow{2}{*}{0} &
                    &
                    \\
                    &
                    &
                    \multirow{2}{*}{$-1$} &
                    &
                    \multirow{2}{*}{0} &
                    \\
                    \multirow{2}{*}{5} &
                    \multirow{2}{*}{$-5$} &
                    &
                    \multirow{2}{*}{0} &
                    &
                    \\
                    &
                    &
                    \multirow{3}{*}{$\vdots$} &
                    &
                    \multirow{3}{*}{$\vdots$} &
                    \\
                    $\vdots$ &
                    $\vdots$ &
                    &
                    $\vdots$ &
                    &
                    \\
                    \multirow{2}{*}{$n-1$} &
                    \multirow{2}{*}{$-(n-1)$} &
                    &
                    \multirow{2}{*}{0} &
                    &
                    \\
                    &
                    &
                    \multirow{2}{*}{$-1$} &
                    &
                    \multirow{2}{*}{$C(n)$} &
                    \\
                    \multirow{2}{*}{n} &
                    \multirow{2}{*}{$-n$} &
                    &
                    \multirow{2}{*}{$B(n)$} &
                    &
                    \\
                    &
                    &
                    \multirow{2}{*}{$A(n)$} &
                    &
                    &
                    \\
                    \multirow{2}{*}{0} &
                    \multirow{2}{*}{$(-1)^n$} &
                    &
                    &
                    &
                    \\
                    &
                    &
                    &
                    &
                    &
                
                \end{tabular}
                \caption{Método das diferenças de Newton para o polinômio $P$.
                Os números em negrito são os coeficientes $c_i$.
                Os números $A(n), \cdots$ são funções de $n$, não necessariamente $0$.}
                \label{tab:diff}
            \end{table}

            Dessa forma, nós concluímos que $c_0 = c_1 = -1$ e que
            $c_2 = \cdots = c_{n-1} = 0$.
            Apenas nos resta encontrar $c_n$, que, seguindo a tabela,
            será uma função de $n$. Até agora, temos
            
            $$P(x) = -1 -1(x-1) + c_n(x-1)\cdots(x-n)$$

            Sabendo que $P(0) = (-1)^n$, vemos que

            \begin{align*}
                P(0) &= (-1)^n \\
                c_n(-1)\cdots(-n) &= (-1)^n \\
                c_n(-1)^n n! &= (-1)^n \\
                c_n &= \dfrac{1}{n!} \\
            \end{align*}

            Tendo todos os coeficientes, podemos calcular $P(k)$, $k$ natural e maior que $n$:

            \begin{align*}
                P(k) &= -1-1(k-1)+\dfrac{1}{n!}(k-1)\cdots(k-n) \\
                &= -k + \dfrac{(k-1)!}{n!(n-1-n)!} \\
                &= {k-1\choose n} - k
            \end{align*}

        \pagebreak

        \item[8.] Vamos montar o polinômio interpolador $p_{k-2}$ de $f$ com os pontos $\{x_1, \cdots, x_{k-1}\}$, através dos métodos das diferenças de Newton:
            
            $$p_{k-2}(x) = c_0 + \cdots + c_{k-2} (x - x_1)\cdots(x - x_{k-2})$$

            Uma das principais características do método de Newton é que podemos ``reutilizar'' o polinômio
            ao adicionar mais pontos. Se quisermos interpolar $\{x_1, \cdots, x_k\}$, basta fazermos
            
            $$p_{k-1}(x) = c_0 + \cdots + c_{k-2} (x - x_1)\cdots(x - x_{k-2}) + c_{k-1} (x - x_1)\cdots(x - x_{k-1})$$

            Observe que $c_{k-1}$ corresponde a $\Delta[x_1, \cdots, x_k]$, que é o que queremos. Segue:

            \begin{align}
                p_{k-1}(x) - p_{k-2}(x) &= c_{k-1} (x - x_1)\cdots(x - x_{k-1}) \nonumber \\
                (f(x) - p_{k-2}(x)) - (f(x) - p_{k-1}(x)) & = c_{k-1} (x - x_1)\cdots(x - x_{k-1}) \label{eq:first}
            \end{align}

            Seja $A = [\min\{x_1, \cdots, x_k\}, \max\{x_1, \cdots, x_k\}]$.
            Por conveniência, assumindo que $\{x_1, \cdots, x_k\}$ são diferentes
            e sabendo que $f = \sin$ é de classe $C^\infty$ no intervalo $A$,
            sabemos, por teorema, que, para cada $x \in A$,
            existem $\alpha_x, \beta_x$
            tais que

            $$f(x) - p_{k-2}(x) = \dfrac{f^{(k-1)}(\alpha_x)}{(k-1)!}(x-x_1)\cdots(x-x_{k-1})$$
            $$f(x) - p_{k-1}(x) = \dfrac{f^{(k)}(\beta_x)}{k!}(x-x_1)\cdots(x-x_k)$$

            Então, se avaliarmos essas funções em $x_k$ e substituirmos na Equação \ref{eq:first}, obtemos

            \begin{align*}
                \dfrac{f^{(k-1)}(\alpha_{x_k})}{(k-1)!}(x_k-x_1)\cdots(x_k-x_{k-1}) &= c_{k-1} (x_k - x_1)\cdots(x_k - x_{k-1}) \\
                \dfrac{f^{(k-1)}(\alpha_{x_k})}{(k-1)!} &= c_{k-1}
            \end{align*}

            Segue portanto:

            \begin{align*}
                |c_{k-1}| &= \left|\dfrac{f^{(k-1)}(\alpha_{x_k})}{(k-1)!}\right| \\
                &= \dfrac{|c^{k-1} \sin(c \alpha_{x_k}+d)|}{(k-1)!} \textrm{ ou } \dfrac{|c^{k-1} \cos(c \alpha_{x_k}+d)|}{(k-1)!} \\
                &\le \dfrac{|c|^{k-1}}{(k-1)!}
            \end{align*}

            Basta agora mostrar que isso vai para zero quando k vai para infinito.
            Ou seja, vamos mostrar que, se $a \ge 0$,

            $$\lim_{k \to \infty} \dfrac{a^k}{k!} = 0$$

            Seja a sequência $y_k = \dfrac{a^k}{k!}$. Considere o termo $y_{2a+1}$. Então:

            $$y_{2a+1} = \dfrac{a^{2a+1}}{(2a+1)!} = \dfrac{a^{2a}}{(2a)!}\dfrac{a}{2a + 1} < \dfrac{a^{2a}}{(2a)!}\dfrac{1}{2} = \dfrac{1}{2} y_{2a}$$

            Concluímos que, para $k$ natural, $y_{2a+k} < \left(\dfrac{1}{2}\right)^k y_{2a}$. Logo:

            $$\lim_{k \to \infty} \dfrac{a^k}{k!} = \lim_{k \to \infty} y_k = \lim_{k \to \infty} y_{2a+k} \le \lim_{k \to \infty} \left(\dfrac{1}{2}\right)^k y_{2a} = 0$$

            Portanto, $|c_{k-1}| \to 0$, ou seja, $\Delta[x_1, \cdots, x_k] \to 0$

    \end{enumerate}
\end{document}
