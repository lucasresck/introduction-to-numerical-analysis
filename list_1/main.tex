\documentclass{article}
\usepackage[T1]{fontenc}
\usepackage[utf8]{inputenc}
\usepackage[portuguese]{babel}

\title{Lista 1 \\
\large Introdução à Análise Numérica \\
Métodos iterativos para sistemas de equações lineares}
\author{Lucas Emanuel Resck Domingues}
\date{Setembro de 2020}

\usepackage{natbib}
\usepackage{graphicx}
\usepackage{amsmath}
\usepackage{listings}
\usepackage{multirow}
\usepackage{hyperref}

\begin{document}

    \maketitle

    \begin{enumerate}
        \item \begin{enumerate}
            \item Vamos mostrar que o método converge.
                Basta que os autovalores sejam menores do que 1.
                Seja $C$ tal que $x^{(n+1)} = Cx^{(n)} + b$. Então segue que:
                \begin{align*}
                    (D+wL)^{-1}((1-w)D-wU)v &= \lambda v \\
                    ((1-w)D-wU)v &= (D+wL)\lambda v
                \end{align*}
                Escolhemos uma linha $i$. Então vemos que:
                \begin{align*}
                    (1-w)a_{ii}v_i - w \sum_{j=i+1}^{m^2} a_{ij}v_j = a_{ii} \lambda v_i + \lambda w \sum_{j=1}^{i-1} a_{ij}v_j \\
                    -a_{ii} \lambda v_i = \lambda w \sum_{j=1}^{i-1} a_{ij}v_j - (1-w)a_{ii}v_i + w \sum_{j=i+1}^{m^2} a_{ij}v_j \\
                    |a_{ii} \lambda v_i| = \left|\lambda w \sum_{j=1}^{i-1} a_{ij}v_j - (1-w)a_{ii}v_i + w \sum_{j=i+1}^{m^2} a_{ij}v_j\right| \\
                    |a_{ii} \lambda v_i| = \left|\lambda w \sum_{j=1}^{i-1} (-a_{ij})v_j + (1-w)a_{ii}v_i + w \sum_{j=i+1}^{m^2} (-a_{ij})v_j\right| \\
                \end{align*}
                Suponhamos que exista $k$ tal que $v_k$ é estritamente maior do que algum $v_i$. Então é óbvio que:
                \begin{align*}
                    |a_{ii} \lambda| &< \left|\lambda w \sum_{j=1}^{i-1} (-a_{ij}) + (1-w)a_{ii} + w \sum_{j=i+1}^{m^2} (-a_{ij})\right| \\
                    |4 \lambda| &< \left|2\lambda w + (1-w)4 + 2w\right| \\
                \end{align*}
                Caso nossa suposição não seja válida, ou seja, todos os componentes de $v$ são iguais, então escolhemos a primeira linha de A
                que possui a soma da linha (com exceção da diagonal), em módulo, menor do que 4. Então vale que:
                \begin{align*}
                    |a_{ii} \lambda| &= \left|\lambda w \sum_{j=1}^{i-1} (-a_{ij}) + (1-w)a_{ii} + w \sum_{j=i+1}^{m^2} (-a_{ij})\right| \\
                    |4 \lambda| &< \left|2\lambda w + (1-w)4 + w 2\right| \\
                \end{align*}
                Em ambos os casos, temos
                \begin{align*}
                    |\lambda| &< \left|\lambda \dfrac{w}{2} - (1-w) + \dfrac{w}{2} \right| \\
                    & \le |\lambda| \left|\dfrac{w}{2}\right| + \left|1 - \dfrac{w}{2}\right| \\
                    |\lambda| &< \dfrac{\left|1 - \dfrac{w}{2}\right|}{\left|1 - \dfrac{w}{2}\right|} \\
                    &= 1
                \end{align*}
                Então o método SOR converge para essa matriz, desde que $0 < w < 2$, para qualquer $m$ natural.

            \item O código do método SOR para essa matriz implementado em MATLAB pode ser conferido no Apêndice \ref{appendix:a}.
            
                A matriz em si não foi implementada, pois seus valores são previsíveis no momento da iteração.
                Isso é bom, pois guardar matrizes tão grandes na memória é muito custoso.
                Além disso, um vetor denso demonstrou maior velocidade em relação a um esparso.
                Mais ainda, para a previsão de qual o valor dos termos da matriz durante a iteração, fez-se necessário
                o cálculo do resto de uma divisão (função \lstinline{mod}), que, ao invés de ser calculada em toda iteração,
                foi calculada para todos os valores e salvas em um vetor antes da iteração. Por mais que utilizamos mais memória,
                para os tamanhos de matrizes utilizadas, valeu a pena a relação custo-benefício.

                Verifiquei, através de alguns experimentos, que o algoritmo não é muito sensível à variação no valor inicial do vetor $x$ ($x^{(0)}$).
                Dessa forma, um vetor inicial razoável é $x^{(0)}=(0, \cdots, 0)$.

                % x_0 = 000000

                Os resultados experimentais podem ser conferidos na Tabela \ref{tab:omega_m}.

                \begin{table}[!h]
                    \centering
                    \begin{tabular}{cc|c|c|c|c|c|c|c|}
                        \cline{3-9}
                                            &               & \multicolumn{7}{c|}{\textbf{$\omega$}}      \\ \cline{3-9} 
                        &
                        &
                        \textbf{1} &
                        \textbf{0.5} &
                        \textbf{0} &
                        \textbf{1.8840*} &
                        \textbf{1.9397*} &
                        \textbf{1.9937*} &
                        \textbf{1.9987*} \\ \hline
                        \multicolumn{1}{|c|}{\multirow{4}{*}{\textbf{$m$}}} &
                        \textbf{50} &
                        3343 &
                        10034 &
                        X &
                        142 &
                        - &
                        - &
                        - \\ \cline{2-9} 
                        \multicolumn{1}{|c|}{} & \textbf{100}  & 13114 & 39348 & X & - & 281 & -    & -     \\ \cline{2-9} 
                        \multicolumn{1}{|c|}{} & \textbf{1000} & 1288257 & $>>$ & X & - & -   & 2787 & -     \\ \cline{2-9} 
                        \multicolumn{1}{|c|}{} & \textbf{5000} & $>>$ & $>>$ & X & - & -   & -    & 13922 \\ \hline
                    \end{tabular}
                    \caption{Valores de $n$ para diversos valores de $m$ e de $\omega$ para o método de SOR. Os valores de $\omega$ com asterisco ($*$)
                    significam os valores ótimos dos respectivos valores de $m$. X significa que o algoritmo não convergiu. $>>$ significa que o algoritmo
                    leva muito tempo (não necessariamente muitas iterações) para ser realizado com os recursos computacionais disponíveis para este trabalho.}
                    \label{tab:omega_m}
                \end{table}

                Fica óbvio que os valores de $\omega$ ótimos para cada $n$ performaram muito melhor, para cada $m$.
                Na verdade, para alguns valores de $n$, como 1000 e 5000, ficou quase impossível realizar os cálculos com meus recursos computacionais para outros valores de $\omega$,
                por mais que na teoria a convergência seja garantida.
                
                Sabemos que, se o algoritmo converge, então 0<$\omega$<2. Então, para $\omega=0$, o
                método não converge, o que foi verificado com a implementação.
                
        \end{enumerate}
    \end{enumerate}

    \appendix

    \section{Código para a questão 1.(b)}
        \label{appendix:a}

        \begin{lstlisting}[language=Matlab]
function b = b_vector(m)
    b = sparse(m^2, 1);
    b(1:m) = 1;
    b((m^2-m+1):m^2) = 1;
    for i=1:m
        b(i*m) = b(i*m) + 1;
        b(i*m-m+1) = b(i*m-m+1) + 1;
    end
    b = b*2;
end

function [x, n, time] = sor(w, m)
    tic
    b = full(b_vector(m));
    m2 = m^2;
    x = zeros(m2, 1);
    err = 1;
    n = 0;
    mod_vector = zeros(m2, 1);
    for i=1:m2
        mod_vector(i) = mod(i, m);
    end
    while err > 10e-6
        for i=1:m2
            sum = 0;
            if i > m
                sum = sum - x(i-m);
            end
            if i <= m2-m
                sum = sum - x(i+m);
            end
            mod_i = mod_vector(i);
            if mod_i ~= 0
                sum = sum - x(i+1);
            end
            if mod_i ~= 1
                sum = sum - x(i-1);
            end
            x(i) = x(i) + w*((b(i)-sum)/4-x(i));
        end
        err = norm(x-2, 'inf');
        n = n + 1;
    end
    time = toc;
end
        \end{lstlisting}
\end{document}
