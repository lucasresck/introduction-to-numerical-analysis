\documentclass{article}
\usepackage[T1]{fontenc}
\usepackage[utf8]{inputenc}
\usepackage[portuguese]{babel}

\title{Lista 1 \\
\large Introdução à Análise Numérica \\
Métodos iterativos para sistemas de equações lineares}
\author{Lucas Emanuel Resck Domingues}
\date{Setembro de 2020}

\usepackage{natbib}
\usepackage{graphicx}
\usepackage{amsmath}

\begin{document}

    \maketitle

    \begin{enumerate}
        \item \begin{enumerate}
            \item Vamos mostrar que o método converge.
                Basta que os autovalores sejam menores do que 1.
                Seja $C$ tal que $x^{(n+1)} = Cx^{(n)} + b$. Então segue que:
                \begin{align*}
                    (D+wL)^{-1}((1-w)D-wU)v &= \lambda v \\
                    ((1-w)D-wU)v &= (D+wL)\lambda v
                \end{align*}
                Escolhemos uma linha $i$. Então vemos que:
                \begin{align*}
                    (1-w)a_{ii}v_i - w \sum_{j=i+1}^{m^2} a_{ij}v_j = a_{ii} \lambda v_i + \lambda w \sum_{j=1}^{i-1} a_{ij}v_j \\
                    -a_{ii} \lambda v_i = \lambda w \sum_{j=1}^{i-1} a_{ij}v_j - (1-w)a_{ii}v_i + w \sum_{j=i+1}^{m^2} a_{ij}v_j \\
                    |a_{ii} \lambda v_i| = \left|\lambda w \sum_{j=1}^{i-1} a_{ij}v_j - (1-w)a_{ii}v_i + w \sum_{j=i+1}^{m^2} a_{ij}v_j\right| \\
                    |a_{ii} \lambda v_i| = \left|\lambda w \sum_{j=1}^{i-1} (-a_{ij})v_j + (1-w)a_{ii}v_i + w \sum_{j=i+1}^{m^2} (-a_{ij})v_j\right| \\
                \end{align*}
                Suponhamos que exista $k$ tal que $v_k$ é estritamente maior do que algum $v_i$. Então é óbvio que:
                \begin{align*}
                    |a_{ii} \lambda| &< \left|\lambda w \sum_{j=1}^{i-1} (-a_{ij}) + (1-w)a_{ii} + w \sum_{j=i+1}^{m^2} (-a_{ij})\right| \\
                    |4 \lambda| &< \left|2\lambda w + (1-w)4 + 2w\right| \\
                \end{align*}
                Caso nossa suposição não seja válida, ou seja, todos os componentes de $v$ são iguais, então escolhemos a primeira linha de A
                que possui a soma da linha (com exceção da diagonal), em módulo, menor do que 4. Então vale que:
                \begin{align*}
                    |a_{ii} \lambda| &= \left|\lambda w \sum_{j=1}^{i-1} (-a_{ij}) + (1-w)a_{ii} + w \sum_{j=i+1}^{m^2} (-a_{ij})\right| \\
                    |4 \lambda| &< \left|2\lambda w + (1-w)4 + w 2\right| \\
                \end{align*}
                Em ambos os casos, temos
                \begin{align*}
                    |\lambda| &< \left|\lambda \dfrac{w}{2} - (1-w) + \dfrac{w}{2} \right| \\
                    & \le |\lambda| \left|\dfrac{w}{2}\right| + \left|1 - \dfrac{w}{2}\right| \\
                    |\lambda| &< \dfrac{\left|1 - \dfrac{w}{2}\right|}{\left|1 - \dfrac{w}{2}\right|} \\
                    &= 1
                \end{align*}
                Então o método SOR converge para essa matriz, desde que $0 < w < 2$, para qualquer $m$ natural.
        \end{enumerate}
    \end{enumerate}

    \bibliographystyle{plain}
\bibliography{references}
\end{document}
